\documentclass[10pt,letterpaper]{article}
\usepackage[utf8]{inputenc}
\usepackage{amsmath}
\usepackage{amsfonts}
\usepackage{amssymb}
\begin{document}
\section*{Demonstration Plan Overview}
\begin{enumerate}
\item Demonstrate communication and parsing of GPS data by entering receiver node calibration mode.
	\begin{enumerate}
	\item The User interfaces with the ARM LCD screen and follows on-screen calibration instructions.
	\item GPS data is received over UART on the mobile node.
	\item GPS data is then parsed and transmitted using MiWi to the receivers.
	\item ARM board queries node0 for GPS position information.
	\item ARM board accumulates several samples of GPS position information and determines the designated node's position.
	\end{enumerate}
\item Demonstrate position estimation algorithm by placing the system into run mode.
	\begin{enumerate}
	\item As the user moves about, GPS data is transmitted from the mobile node.
	\item Each receiver node calculates received signal strength from the mobile node.
	\item node1 and node2 forward their received signal strength to node0.
	\item ARM board queries node0 for all RSSI and GPS data.
	\item Positioning algorithm updates position estimation based on RSSI data.
	\end{enumerate}
\item Demonstrate web interface functionality
	\begin{enumerate}
	\item After section 2 is complete, continue to operate system in run mode.
	\item Establish a connection to the ARM webserver.
	\item Recent position information will be displayed on the web page.
	\item The web page refresh rate is once every 10 seconds.
	\end{enumerate}
\item Demonstrate filesystem functionality
	\begin{enumerate}
	\item After sections 2 and 3 are complete, shut down the ARM board.
	\item Remove the SD card and insert into a computer.
	\item Display the logged information on a computer.
	\end{enumerate}
\item Demonstration Complete.
\end{enumerate}
\end{document}